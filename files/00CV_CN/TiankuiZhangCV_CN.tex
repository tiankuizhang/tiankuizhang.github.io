% !TEX TS-program = xelatex
% !TEX encoding = UTF-8 Unicode
% !Mode:: "TeX:UTF-8"

\documentclass{resume}
\usepackage{zh_CN-Adobefonts_external} % Simplified Chinese Support using external fonts (./fonts/zh_CN-Adobe/)
% \usepackage{NotoSansSC_external}
% \usepackage{NotoSerifCJKsc_external}
% \usepackage{zh_CN-Adobefonts_internal} % Simplified Chinese Support using system fonts
\usepackage{linespacing_fix} % disable extra space before next section
\usepackage{cite}

\begin{document}
\pagenumbering{gobble} % suppress displaying page number

\name{张天奎}

\basicInfo{
  \email{zhang.tiankui@foxmail.com} \textperiodcentered\ 
 % \phone{(+1) 5209894134} \textperiodcentered\ 
 % \linkedin[tiankui]{https://www.linkedin.com/in/tiankuizhang/}
 	\homepage[https://tiankuizhang.github.io]{https://tiankuizhang.github.io/}
 	%\github[Github]{https://github.com/tiankuizhang}
 	}
 
\section{\faGraduationCap\  教育背景}
\datedsubsection{\textbf{亚利桑那大学}, 图森, 亚利桑那州, 美国}{2014年8月 -- 2020年7月}
\textit{博士}\ 计算物理, GPA: 4.00/4.00
\datedsubsection{\textbf{武汉大学},武汉,湖北}{2010年9月 -- 2014年6月}
\textit{学士}\ 物理学, GPA: 3.85/4.00
\datedsubsection{\textbf{伦敦大学国王学院},伦敦,英国}{2013年9 -- 2014年5月}
\textit{国际交换生}\ 物理学

\section{\faFlask\ 工作/研究经历}
\datedsubsection{\textbf{泊松软件},上海}{2025年2月 -- 至今}
\role{开发工程师}{维护几何引擎}
\begin{onehalfspacing}
\begin{itemize}
  \item 冗余拓扑,拔模
\end{itemize}
\end{onehalfspacing}

\datedsubsection{\textbf{新迪},上海}{2023年6月 -- 2025年1月}
\role{资深研究员}{给CAD软件添加特征}
\begin{onehalfspacing}
\begin{itemize}
  \item 偏置曲线,桥接曲面,曲面光顺,拟合曲线,边界混合,拔模偏移
\end{itemize}
\end{onehalfspacing}


\datedsubsection{\textbf{广联达},上海}{2020年9月 -- 2023年6月}
\role{几何算法研发工程师}{针对上层业务需求开发几何算法}
\begin{onehalfspacing}
\begin{itemize}
  \item brep体的自适应水密离散
\end{itemize}
\end{onehalfspacing}


\datedsubsection{\textbf{亚利桑那大学},图森}{2014年8月 -- 2020年5月}

\role{研究助理}{使用能量极小曲面研究细胞膜的动力学变形,导师 Charles Wolgemuth}
\begin{onehalfspacing}

\begin{itemize}
  \item 计算能量最小曲面的动力学过程
  \item 使用微分几何和变分法计算Helfrich曲面的弹性动力模型
  \item 为含有level set-defined的边界条件的Hamilton-Jacobi方程设计并实现了一个6阶精度的数值算法
  \item 用Matlab和CUDA C++设计并实现了模拟单相膜,双相膜,以及蛋白质-膜相互作用的并行数值算法
\end{itemize}
\end{onehalfspacing}





%\datedsubsection{\textbf{\LaTeX\ 简历模板}}{2015 年5月 -- 至今}
%\role{\LaTeX, Python}{个人项目}
%\begin{onehalfspacing}
%优雅的 \LaTeX\ 简历模板, https://github.com/billryan/resume
%\begin{itemize}
%  \item 容易定制和扩展
%  \item 完善的 Unicode 字体支持,使用 \XeLaTeX\ 编译
%  \item 支持 FontAwesome 4.5.0
%\end{itemize}
%\end{onehalfspacing}

% Reference Test
%\datedsubsection{\textbf{Paper Title\cite{zaharia2012resilient}}}{May. 2015}
%An xxx optimized for xxx\cite{verma2015large}
%\begin{itemize}
%  \item main contribution
%\end{itemize}

\section{\faCogs\ 技能}
% increase linespacing [parsep=0.5ex]
\begin{itemize}[parsep=0.5ex]
  \item 编程语言: C++,Matlab,CUDA
  \item PDE相关数值算法: 水平集,有限体积
  \item 计算几何:实体建模,nurbs曲线曲面,线性代数及相关优化算法
  \item 理论背景: 物理,微分几何,应用数学
\end{itemize}

%\section{\faInfo\ 论文}
\section{\faBook\ 论文}
% increase linespacing [parsep=0.5ex]
\begin{itemize}[parsep=0.5ex]
  \item \textbf{Tiankui Zhang} and Charles Wolgemuth. Sixth-order accurate schemes for reinitialization and extrapolation in the level set framework. \textit{Journal of Scientific Computing}, 83(2), 2020
  \item \textbf{Tiankui Zhang} and Charles Wolgemuth. A general computational framework for the dynamics of single- and multi-phase vesicles and membranes. \textit{Journal of Computational Physics}, Volume 450, 2022, 110815, ISSN 0021-9991
\end{itemize}

%\section{\faThumbsUp\ 荣誉}
%\datedline{图形平台图形之星, 广联达}{2020}
%\datedline{College of Science Galileo Circle Scholarship,亚利桑那大学}{2019}
%\datedline{弘毅学堂荣誉毕业生,武汉大学}{2014}
%\datedline{留学基金委交换生奖学金,武汉大学}{2013}

%\section{\faInfo\ 论文}
% increase linespacing [parsep=0.5ex]
%\begin{itemize}[parsep=0.5ex]
%  \item 技术博客: http://blog.yours.me
%  \item GitHub: https://github.com/username
%  \item 语言: 英语 - 熟练(TOEFL xxx)
%\end{itemize}

%% Reference
%\newpage
%\bibliographystyle{IEEETran}
%\bibliography{mycite}
\end{document}
